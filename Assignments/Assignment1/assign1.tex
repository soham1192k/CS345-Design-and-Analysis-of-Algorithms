\documentclass{article}

% If you're new to LaTeX, here's some short tutorials:
% https://www.overleaf.com/learn/latex/Learn_LaTeX_in_30_minutes
% https://en.wikibooks.org/wiki/LaTeX/Basics

% Formatting
\usepackage[utf8]{inputenc}
\usepackage[margin=1in]{geometry}
\usepackage[titletoc,title]{appendix}
\def\code#1{\texttt{#1}}
\usepackage{listings}
\usepackage{color}
\usepackage[linesnumbered,ruled,vlined]{algorithm2e}
\newcommand{\forcond}{$i=0$ \KwTo $n$}
\SetKwProg{Fn}{Function}{}{end}\SetKwFunction{FRecurs}{fasterNDP}%
\SetAlgoLongEnd

\definecolor{dkgreen}{rgb}{0,0.6,0}
\definecolor{gray}{rgb}{0.5,0.5,0.5}
\definecolor{mauve}{rgb}{0.58,0,0.82}

\lstset{frame=tb,
  language=C++,
  aboveskip=3mm,
  belowskip=3mm,
  showstringspaces=false,
  columns=flexible,
  basicstyle={\small\ttfamily},
  numbers=left,
  stepnumber=1,
  numberstyle=\tiny\color{gray},
  keywordstyle=\color{blue},
  commentstyle=\color{dkgreen},
  stringstyle=\color{mauve},
  breaklines=true,
  breakatwhitespace=true,
  tabsize=3,
  frame=shadowbox,
  rulesepcolor=\color{gray}
}
% Math
% https://www.overleaf.com/learn/latex/Mathematical_expressions
% https://en.wikibooks.org/wiki/LaTeX/Mathematics
\usepackage{amsmath,amsfonts,amssymb,mathtools}

% Images
% https://www.overleaf.com/learn/latex/Inserting_Images
% https://en.wikibooks.org/wiki/LaTeX/Floats,_Figures_and_Captions
\usepackage{graphicx,float}
\usepackage{hyperref}
% Tables
% https://www.overleaf.com/learn/latex/Tables
% https://en.wikibooks.org/wiki/LaTeX/Tables

% Algorithms
% https://www.overleaf.com/learn/latex/algorithms
% https://en.wikibooks.org/wiki/LaTeX/Algorithms
\usepackage[ruled,vlined]{algorithm2e}
\usepackage{algorithmic}

% Code syntax highlighting
% https://www.overleaf.com/learn/latex/Code_Highlighting_with_minted
\usepackage{minted}
\usemintedstyle{borland}
\usepackage[T1]{fontenc}
% References
% https://www.overleaf.com/learn/latex/Bibliography_management_in_LaTeX
% https://en.wikibooks.org/wiki/LaTeX/Bibliography_Management
\usepackage{biblatex}
\addbibresource{references.bib}

% Title content
\title{CS345 Assignment-1}
\author{Ayush Shakya \\180178 \and Soham Ghosal \\180771}
\date{August 20, 2021}

\begin{document}

\maketitle
% intro/ objectives 
% sketch of algorithm (Assumptions in bold)
% pseudocode (template dhundo)
% proof of correctness
% time complexity analysis

% Introduction and Objectives
\section{Faster algorithm for Non-Dominated Points in plane}
In the lecture two approaches were discussed to evaluate non-dominated points in a plane. The first approach was input dependent and worked in a time-complexity of $O(N*h)$ where h is the number of dominated points in the \textbf{plane consisting of N points}. This algorithm was $O(N)$ in best case but $O(N^2)$ in the worst case.
\par
We then improved this approach to $O(NlogN)$. This new algorithm was based on \textbf{divide-and-conquer paradigm}, but there were scenarios where the first algorithm could outperform the second algorithm. In this assignment, we devise an algorithm with a time complexity of $O(Nlogh)$ which is faster than both these algorithms, specifically in cases where the number of non-dominated points are very few. As mentioned in the problem statement, it can be shown that if n points are selected randomly uniformly from a unit square, then the expected(average) number of non-dominated points is just $O(log n)$.

%  Sketch of Algorithm
\section{Sketch of Algorithm}

\subsection{Assumptions}
\begin{itemize}
    \item We assume that \textbf{no two points in the plane have same x-coordinate or the same y-coordinate}.
    \item There will be only one point on the vertical line passing through the median of x- coordinate of points.
    \item We define the x-median to be the median of x coordinates of all given points-in case there are even number of points, we take the floor of the actual median.
    \item The set of points S received as an argument to the function is has a non-zero size.
\end{itemize}

\subsection{Intuition}
We begin by computing the x-median of the input points, and splitting the given sample space into two halves, via a vertical line passing through the x-median\textbf{ [Divide Step]}.The median can be computed using an $O(N)$ algorithm, as discussed in class. Since we are splitting on the basis of x-median, the sizes of the two halves would differ by at most one. Let's call these partitions \textbf{left set(LS)} and \textbf{right set(RS)}.

We invoke a recursive call to the divide function for the right set. Among all those points, we choose the point with maximum value of y-coordinate. Let the y-coordinate of this point be $y_{max}$.
\par
We collect all the points in the left set whose y-coordinate is greater than $y_{max}$. Unless this set is empty, we send this set into another recursive call.

%ROUGH

%ROUGH







% Pseudocode
%O(N) complexity for all 3->mention
\newpage
\section{Pseudocode}

% \begin{algorithm}
% 	\caption{Faster algorithm to find NDP in plane} 
% 	\begin{algorithmic}[1]
% 	    \Function {fasterNDP}{S}
% 	        \State \var{xm} \gets \func{getmaxx} (S), \var{ym} \gets \func{getmaxy}(S)
% 	        \State pmax \gets (xm,ym)
%             \State \If {p_{max} \in s} 
%             \State \Return %$p_{max}$
%             \EndIf
% 	    \EndFunction
% 	\end{algorithmic} 
% \end{algorithm}

% \begin{algorithm}[H]
% \DontPrintSemicolon
%     \Function{fasterNDP}{$S$}
%         \State \var{xm} \gets \func{getmaxx} (S), \var{ym} \gets \func{getmaxy}(S)
%         \State pmax \gets (xm,ym)
%         \State \If {p_{max} \in s} 
%     \EndFunction
%     \Function{Increment}{$a$}
%     \State $a \gets a+1$
%     \State \Return $a$
% \EndFunction
% $t=0$\\
% $w_{2}^{0}=w_{0}$\\
% \While{not converged}
% {
% $w_{1}^{(t+1)}=argmin_{w1}\cL(w_{1},w_{2}^{(t)})$\\
% $w_{2}^{(t+1)}=argmin_{w2}\cL(w_{1}^{(t+1)},w_{2}$\\
% $t=t+1$
% }
% \textbf{end}
% \end{algorithm}



\begin{algorithm}
\Fn(){\FRecurs{S}}
{
% \KwData{Some input data\\these inputs can be displayed on several lines and one
% input can be wider than line’s width.}
% \KwResult{Same for output data}
% \tcc{this is a comment to tell you that we will now really start code}
    $x_{m}$ \gets $getmax_x(S)$, $y_m$ \gets $getmax_y(S)$ 
    \\
    $p_{max}$ \gets $(x_m, y_m)$
    \\
    finalNDP \gets \emptyset
    \\
    \If(){$p_{max} \in S$}
    {
        finalNDP.push($p_{max}$)\\
        \Return finalNDP
    }
    \\
    
    (LS,RS) \gets SplitByMedian(S)
    \\
    rightNDP \gets fasterNDP(RS)\\
    $y_{max}$ \gets $getmax_y$(rightNDP)\\
    $reduced_{LS}$ \gets \emptyset
    \\
    
    \For(){point p $\in$ LS}
    {
        \If(){p.y $>$ $y_{max}$}
        {
            $reduced_{LS}$.push(p)
        }
    }
    
    finalNDP.push(rightNDP)
    
    \If() {$reduced_{LS}.size$ $>$ 0} {
        leftNDP \gets fasterNDP($reduced_{LS}$)\\
        finalNDP.push(leftNDP)
    }
    
    \Return finalNDP


}
\caption{Faster Algorithm for non-dominated points}
\end{algorithm}

%ujefuierguiegre
% \begin{algorithm}
% \Fn(){\FRecurs{S}}{
% % \KwData{Some input data\\these inputs can be displayed on several lines and one
% % input can be wider than line’s width.}
% % \KwResult{Same for output data}
% % \tcc{this is a comment to tell you that we will now really start code}
    
%     $x_{m}$ \gets $getmax_x(S)$, $y_m$ \gets $getmax_y(S)$ 
%     \\
%     $p_{max}$ \gets $(x_m, y_m)$
%     \\
%     finalNDP \gets \emptyset
%     \\
    
%     \If(){$p_{max} \in S$}{
%          finalNDP.push($p_{max}$)\\
%         \Return finalNDP
%     }
    
%     (LS,RS) \gets SplitByMedian(S)
%     \\
%     rightNDP \gets fasterNDP(RS)\\
%     $y_{max}$ \gets $getmax_y$(rightNDP)\\
%     $reduced_{LS}$ \gets \emptyset
%     \\
    
%     \For(){point $p \in LS$}
%     {
%         \If(){p.y $> y_{max}$}
%         {
%             $reduced_{LS}$.push(p)
%         }
%     }
    
%     finalNDP.push(rightNDP)
    
%     % \tcc{now loops}
%     % \For{\forcond}{
%     % a for loop\;
%     % }

%     \If(){$reduced_{LS}$.size $>$ 0}{
%         leftNDP \gets fasterNDP($reduced_{LS}$)\\
%         finalNDP.push(leftNDP)
%     }

%     finalNDP.push(rightNDP)
    
%     % \If() {reduced_{LS}.size $>$ 0} {
%     %     leftNDP \gets fasterNDP(reduced_{LS})\\
%     %     finalNDP.push(leftNDP)
%     % }
    
%     \Return finalNDP

% }
% \caption{Faster Algorithm for non-dominated points}
% \end{algorithm}
%nrgirgirg

%\vspace{-1cm}

\section {Proof of Correctness}

First we consider the base case. Since we assumed that x-coordinate and y-coordinate of all points are unique, if there exists a point whose both x-coordinate and y-coordinate are maximum among all points in that current set, it dominates all other points in that set(since, it encompasses all other points). Hence it is the only non-dominated point in the given set and we return it as an answer and exit from the recursive call.
\par
Next, we need to prove that set of points $rightNDP$ obtained in recursive call on right set(RS) are non-dominated points. Since their x-coordinates are greater than all the points in the left set(LS), they are non-dominated with respect to the points in the left set, and we already know, they are non-dominated among themselves. Thus we add $rightNDP$ into the final list of non-dominated points.
\par
For the left part, we construct a set of points $reduced_{LS}$ whose y-coordinate is greater than $y_{max}$. This means these points are not dominated by any point in right set(RS). After a recursive call on this set, we get a set of non-dominated points $leftNDP$ among the $reduced_{LS}$. We already proved that these points are not dominated by right set(RS) due to the condition that their y-coordinate is greater than $y_{max}$, and after the recursive call, we know that they are non-dominated among themselves. Hence, this set of points also holds the non-dominated conditions for the current set S. Thus we add $leftNDP$ into the final list of non-dominated points and return this as an answer.
\newpage
\section {Time Complexity Analysis}

We try to analyse the time complexity using induction. Let T(h) denote the time-complexity of the proposed function when there are exactly "h" non-dominated points among the current set of points S, of size N. The functions $getmax_{x}$, $getmax_{y}$ and $p_{max}$ checking condition,  x-median calculation and splitting in LS and RS, filtering points on basis of $y_{max}$ and add to final list are linear time functions in N. So, for cases where h=1, the time complexity is O(N) (Since log(1)=0).\\Note that, after division of the sample space into two halves, there might be an arbitrary number of non-dominated points in the left half(let's call this $h_{1}$), and as a consequence, the number of non-dominated points in the right half would be $h-h_{1}$. Keeping this in mind, we would have to recur on the left half(containing n/2 points) and the right half(containing n/2 points).Thus the time complexity when there are h non-dominated points would be bounded by the maximum of all such divisions, hence the recurrence relation (equation 2).
\vspace{2mm}\\The inductive hypothesis is assumed to be as follows :
\begin{center}
    \begin{equation}
    \textbf{Induction Hypothesis} : \framebox[1.1\width]{T(h) \leq  c*(N + N*log_2(h))} \hspace{2mm}\textbf{for N points}.
    \end{equation}
\end{center}
\begin{flushleft}
When left set(LS) is completely deleted by $y_{max}$, the size of h is 0. For such a case, T(h) for any size of N is 0. For the case of h=1, the algorithm is simply linear time due to the $p_{max}$ if condition.
\end{flushleft}
The recurrence relation for this pseudocode can be written as :
\begin{center}
\begin{equation}
        \framebox[1.1\width]{T(h) \leq  max ( T(h_1) + T(h-h_1) + c_1*N ) \hspace{2mm}\forall\hspace{2mm}{0 \leq h_{1} \leq h} }
\end{equation}

\end{center}
Please note that T(h) is defined for all N points, but T($h_{1}$), T($h-h_{1}$) are defined for N/2 points. Value of $h_{1}$ can vary from 0 to N, thus, we need to take the max of all the right recurrences and compare it with the T(h) for N points.

\begin{center}
\begin{equation}
        \framebox[1.1\width]{T(h_1) \leq c*(N/2 + N/2*log_2(h_1)) \hspace{2mm}\text{and} \hspace{2mm}T(h-h_1) \leq c*(N/2 + N/2*log_2(h_1))}
\end{equation}
\end{center}

Let us substitute these induction hypothesis for N/2 points into equation 2.
\begin{center}
        T(h) \leq  $max ( T(h_1) + T(h-h_1) + c_1*N ) \hspace{2mm}\forall\hspace{2mm}{0 \leq h_{1} \leq h}$ \vspace{2mm}\\\textbf{=>}T(h) \leq $ max ( c*(N/2 + N/2*log_2(h_1) + c*(N/2 + N/2*log_2(h_1)) + c_1*N ) \hspace{2mm}\forall\hspace{2mm}{0 \leq h_{1} \leq h} $
        \vspace{2mm}\\\textbf{=>}T(h) \leq $ max ( (c+c_1)*N + c*N/2*(log_2(h_1*(h-h_1))) ) \hspace{2mm}\forall\hspace{2mm}{0 \leq h_{1} \leq h} $
\end{center}









The ($c+c_{1}$) component is independent of $h_{1}$. It can be easily shown through differentiation that the max value of $h_{1} * h-h_{1}$ is achieved at $h_{1} = h/2$, which gives the value of $h_{1} * h-h_{1} = h^{2}/4$. If h is an odd number, then too the value of $h_{1} * h-h_{1} < h^{2}/4$. Thus, we can say that, $h_{1} * h-h_{1} \leq h^{2}/4$. $log_2(x)$ is a monotonically increasing function in x (input). So we can say: \vspace{3mm}
\begin{center}
T(h) \leq $ max ( (c+c_1)*N + c*N/2*(log_2(h^2/4))  $
\vspace{2mm}\\
=>T(h) \leq $ max ( (c+c_1)*N + c*N*(log_2(h/2))  $
\vspace{2mm}\\
=>T(h) \leq $ max ( (c+c_1)*N + c*N*(log_2(h)-1))  $
\vspace{2mm}\\
=>T(h) \leq $ max ( c_1*N + c*N*(log_2(h)))  $
\end{center}

We can always choose a \textbf{c which is greater than $c_{1}$}. On doing so, we get the desired bound:
\begin{center}
\framebox[1.1\width]{T(h) \leq  c*N(1 + log_2(h)) \hspace{2mm} \text{where c\geq $c_{1}$}}
\\
\vspace{3mm}
Thus, our initial hypothesis is true; the time complexity of the proposed algorithm is $\textbf{O(N*log h)}$. 
\end{center}
\end{document}
